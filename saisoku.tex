\documentclass{jlreq}
\usepackage{graphicx,tcolorbox,diffcoeff}
\usepackage{amsmath,amsthm}
\usepackage{unicode-math}
\setmathfont{NewComputerModernMath}
\newtcolorbox{simplebox}[2]{colframe=black,colback=white,coltitle=black,colbacktitle=white,boxrule=0.8pt,arc=0mm,title={\textbf{#2}},}
\usepackage[thicklines]{cancel}
\usepackage{hyperref}

\begin{document}

時間はエネルギー保存から
\( \int_{t_i}^{t_f}\sqrt{2+y'^2}(2gy)^{-\frac{1}{2} } \dl{x}\) 
これは
\( L=\sqrt{2+y'^2}(y)^{-\frac{1}{2} } \)としてE-Leqをとけばよい

\begin{align}
  \diffp{L}{y}=&-\frac{1}{2}(2+y'^2)^{\frac{1}{2}}y^{-\frac{3}{2}} \notag\\
  \diffp{L}{y'}=&\frac{1}{2} \cdot 2 y'(2+y'^2)y^{-\frac{1}{2} }  =y'(2+y'^2)y^{-\frac{1}{2} } \notag
\end{align}
以降煩雑さ回避のために \( p:= \difs{y}{x} \)を用いる。 

\begin{align*}
  \diffp{}{p} \sqrt{2+p^2}p
  =&\sqrt{2+p^2}- \cancel{\frac{1}{2} }(2+p^2)^{-\frac{3}{2} }p \cdot \cancel{2}p\\
  =&\frac{2}{\sqrt{2+p^2}}\left( 2- \frac{p^2}{2+p^2}  \right)\\
  =&(2+p^2)^{-\frac{3}{2} }   
\end{align*}
これを次の微分で使う

\begin{align*}
  \diff**x{ \diffp{L}{p}}
  =& p \diff{}{y}\diffp{L}{p}\\
  =& py^{-\frac{1}{2} }(2+p^2)^{-\frac{3}{2} }  \diff{p}{y} - \frac{1}{2}p^2y^{-\frac{3}{2} }(1+p^2)^{-\frac{1}{2} }   
\end{align*}
こいつをE-Leqにぶち込む

\begin{align*}
  &py^{-\frac{1}{2} }(1+p^2)^{-\frac{3}{2} } \diff{p}{y}-\frac{1}{2} p^2 y^{-\frac{3}{2} }(1+p^2)^{-\frac{1}{2} }+\frac{1}{2}  y^{-\frac{3}{2} }(1+p^2)^{\frac{1}{2} }=0\\
  &2py(1+p^2)^{-2} \diff{p}{y} - \frac{p^2}{1+p^2}+1=0 \\
  &\frac{2p}{1+p^2} \dl{p} =- \frac{ \dl{y}}{y} \\
  & \ln (y(1+p^2))=\text{const.}\\
  & y(1+p^2)=C
\end{align*}

\end{document}