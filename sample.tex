\documentclass{beamer}
\usepackage{luatexja}
\usetheme[block=fill]{metropolis}
\usepackage{graphicx,diffcoeff}
\usepackage{unicode-math}
\setmathfont{NewComputerModernMath}
\usepackage[thicklines]{cancel}
\usepackage{xcolor}
\renewcommand{\CancelColor}{\color{red!60!black}}
\usepackage{hyperref}
%ここまで呪文

\title{解析力学自主ゼミ第◯回\\
 ~◯◯~}
\author{}
\date{ゼミをやる日}
\begin{document}

%一スライドは一つのframe環境に対応します。
\begin{frame}
  \maketitle
\end{frame}

%これで一枚のスライドを出力します。
\section{〇〇}

%タイトルを付けたい場合はこのように
\begin{frame}{タイトル}
  ◯◯
  \begin{equation}
    ∇ \times (∇ \times \symbfit{A} ) = ∇(∇\cdot \symbfit{A} ) - (∇ \cdot ∇)\symbfit{A} 
    \label{veq}
  \end{equation}%frame環境の中では普通にかける
  式\eqref{veq}より……\alert{強調する箇所}
\end{frame}

\begin{frame}
    % 通常のブロック
    \begin{block}{タイトル}
        〇〇
    \end{block}

    % 警告・注目用の朱色ブロック
    \begin{alertblock}{タイトル}
        〇〇
    \end{alertblock}

    % 例題用のブロック(緑)
    \begin{exampleblock}{タイトル}
        〇〇
    \end{exampleblock}
\end{frame}

\begin{frame}[standout]%色反転
    こんにちは!! \\
    \uncover<2->{こんばんは}\\%二枚目以降表示
    \uncover<1-2>{こんばんは}\\%一から二枚目まで表示
    \only<3>{おはよう}\\%三枚目だけ表示り
\end{frame}

\end{document}

