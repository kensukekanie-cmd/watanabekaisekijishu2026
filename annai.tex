\documentclass{jlreq}
\usepackage{hyperref}
\pagestyle{emptyj}
\begin{document}
\begin{flushright}
  \today
\end{flushright}

\begin{flushleft}
  自主ゼミメンバー
\end{flushleft}

\begin{flushright}
 理学部物理学科1年 蟹江健介
\end{flushright}

\vspace{1\baselineskip}

\begin{center}
  \LARGE 解析力学自主ゼミ計画
\end{center}
\vspace{1\baselineskip}

今回の自主ゼミは春休み期間で行うので、
8週間となります。
かなりタイトなスケジュールゆえに、
演習問題をすべて解説する時間がないと思うので、
各発表者は演習問題の中から絞って紹介してください。

\begin{description}
  \item[第一週] 2/7 第1章
  \item[第二週] 2/14 第2章
  \item[第三週] 2/18,2/21 第3章、第4.1-4.4章
  \item[第四週] 2/25,2/28 第4.5-4.7章
  \item[第五週] 3/7 第5章
  \item[第六週] 3/14 第6章
  \item[第七週] 3/18,3/21 第7章
  \item[第八週] 3/25,3/28 第8章
\end{description}

対面なら図書館、閉館なら蟹江の家(おそらく)。
春休みなので参加できないときはzoomで参加か、
zoomの録画を視聴してください。

授業資料は、GitHubで\TeX ファイルで管理します。
ただし手書き資料をPDF化して共有でも良いです。
\href{https://github.com/kensukekanie-cmd/watanabekaisekijishu2026}{\textbf{授業資料のGitHubリポジドリ}}


\end{document}