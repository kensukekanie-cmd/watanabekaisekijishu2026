\documentclass{jlreq}
\usepackage{graphicx,tcolorbox,diffcoeff}
\usepackage{amsmath,amsthm}
\usepackage{unicode-math}
\setmathfont{NewComputerModernMath}
\usepackage[thicklines]{cancel}
\usepackage{hyperref}
%使いたいパッケージは競合してエラー置きない限り自由に追加していいよ
\tcbuselibrary{breakable, skins}
\newcommand{\qsep}{%
  \tcblower
  \medskip
}


\newtcolorbox{qbox}[2]{%
  breakable,
  enhanced,
  colback=white,
  colframe=red,
  boxrule=0.8pt,
  arc=2mm,
  left=4mm,
  right=4mm,
  top=3mm,
  bottom=3mm,
  before upper={
    \textbf{#1}
    \medskip
  },
  after upper={
    \par\smallskip
    \hfill{\small 質問者:#2}
  }
}

\newtcolorbox{qboxok}[3]{%
  breakable,
  enhanced,
  colback=white,
  colframe=black,
  boxrule=0.8pt,
  arc=2mm,
  left=4mm,
  right=4mm,
  top=3mm,
  bottom=3mm,
  before upper={
    \textbf{#1}
    \medskip
  },
  after upper={
    \par\smallskip
    \hfill{\small 質問者:#2}
  },
  after lower={
    \par\smallskip
    \hfill{\small 回答者:#3}
  }
}


\begin{document}
% 以下のように書いてください

%--解決していない質問--
% \begin{qbox}
%   {教科書該当箇所}
%   {質問者}
  
%   質問文…
% \end{qbox}

%--解決している質問--
% \begin{qboxok}
%   {教科書該当箇所またはタイトル}
%   {質問者}
%   {回答者}
  
%   質問文…

%   \qsep

%   回答文…
% \end{qboxok}


\begin{qboxok}
  {p29  \( t \)に陽に依存しないラグランジアン}
  {蟹江}
  {蟹江}
  
  「ラグランジアンが\( t \)に陽に依存しない」ってどういう状態かよくわからない.

  \qsep

  \( q , \dot{q}\)が定まったときに,ラグランジアンが時間によらず一定の値を取るということ.
  \( q, \dot{q} \) は測定できるから,二次関数のような具体的な関数形が与えられていなくても
  陽に依存していることを定義できる.

\end{qboxok}

\begin{qboxok}
  {最速降下曲線の変形}
  {昌詩(ゼミ中)}
  {蟹江}

  \( p=\tan θ/2 \)という変形をしているが,
  \( p= \difs yx \)なのに勝手にそんなことしていいのか. 

  \qsep

  \( x ,y\)はそれぞれ \( p \)の関数として導くことができました.
  この時点で \( p \)が \( p= \difs yx \)という表式を持っているだけで,媒介変数表示になっており,サイクロイドを表せているわけです.
  たまたま媒介変数表示のパラメーターにわかりやすい性質があっただけで,媒介変数であることに変わりはないので,
  \( p \)の動く範囲がどうであっても全単射な関係を与えられる \( p=\tan θ/2 \)という変換をしていいのです.
\end{qboxok}

\begin{qboxok}
  { \( q_i \mapsto Q_i \)は全単射か }
  {蟹江}
  {全員}

  2次元極座標において \( r=0 \) としたとき,
  任意の \( θ \) で同じ座標に移ってしまうが問題ないのかな.
  
  \qsep

  高々一つの点が不都合が生まれるだけなのだから,
  \( (r,θ) = (0,0) \) の組のみ認めることにすれば良い.

\end{qboxok}

\begin{qboxok}
  {強い意味での作用反作用の法則とポテンシャル}
  {昌詩}
  {全員}

  強い意味の作用反作用の法則が成立することと,
  ポテンシャルの関数が特殊直交行列による変換に対して不変が同値なのがどういうことかわからない.

  \qsep

  強い意味での作用反作用の法則が成立するということは,
  \( \symbfit{r}_i - \symbfit{r}_j \propto \symbfit{f}_{ij} \)
  ということである.
  一般の関数 \( g \) について
  \begin{align*}
    ∇ g(|\symbfit{r}|^2) 
    =& \diff{g(ξ^2)}{ξ}\Bigg|_{ξ=x^2+y^2+z^2} (x \symbfit{e}_x + y \symbfit{e}_y + z \symbfit{e}_z )\\
    =& \diff{g(ξ^2)}{ξ}\Bigg|_{ξ=|\symbfit{r}|^2} \symbfit{r} \\
    \propto & \symbfit{r} 
  \end{align*}
  が成立することから,
  ポテンシャルが 
  \( |\symbfit{r} _i - \symbfit{r} _j | \) の関数になるときに強い意味での作用反作用の法則が成立する.
  これは粒子の相対距離についてのみ依存するということであり,
  座標系の回転によらないことと同値である.
  以上より,
  特殊直交行列は座標を回転のみ行うため疑問であった同値が示される.
  補足として,直交行列ではなく特殊直交行列である理由について述べる.
  直交行列は単に回転操作するだけでなく,空間の反転操作(例:右手座標系 → 左手座標系)
  も含む.物理において回転不変性を議論する場合,ネーターの定理を考えることから連続的な変化を想定する.
  連続な回転では鏡像操作を実現できない不連続な操作であることから特殊直交行列を考えている.
\end{qboxok}

\begin{qboxok}
  {p61 拘束条件の独立}
  {蟹江}
  {蟹江}
   
  拘束条件が独立であることが,
  \[[\tilde{J}(q)]_{γ i} := \diffp{C_{γ}(q)}{q_i}\]
  で与えられる \( Γ \) 行 \( N \) 列行列の階数が \( Γ \) と定義されているが,
  その気持ちがわからない.

  \qsep

  \[0= C_γ (q + ε) - C_γ(q) = \sum_{i=1}^{N} \diffp{C_γ(q)}{q_i} ε_i + 𝒪(ε^2)\]
  を
  \[
    \symbfit{ε} :=
    \begin{pmatrix}
      ε_1\\
      \vdots\\
      ε_N
    \end{pmatrix}
  \]
  用いて書き換えると
  \[0= C_γ (q + ε) - C_γ(q) =  ∇_q C_γ(q) \cdot \symbfit{ε} + 𝒪(ε^2) \]
  すなわち \( ∇_q C_γ(q) \cdot \symbfit{ε} =0 \) を得る.
  \( ∇_q C_γ(q) \) と \( \symbfit{ε} \) が直交するため
  \( ∇_q C_γ(q) \) の方向には \( q \) は動かせない.
  よって 
  \( ∇_q C_γ(q) \) が \( γ =1, \ldots, Γ \) で線形独立になることは
  \( q \)が \( Γ \)  個の方向へ動けなることを示す.
  したがって \( \mathrm{rank} \tilde{J}(q) = Γ\)であるときの 自由度は 
  \( N - Γ \) となる.
\end{qboxok}

\begin{qboxok}
  {p64 (3.19)}
  {全員}
  {太駆}

  ここの変形なにしているのかわからない

  \qsep

  \begin{align*}
    &\sum_{i=1}^{N} \underbrace{\diffp{C_γ(q)}{q_i}}_{=[J(q)]_{γl}}  ε_i =0\\
    \implies & \sum_{l=N'+1}^{N} [J(q)]_{γl} ε_l +\sum_{k=1}^{N'} \diffp{C_γ(q)}{q_k} ε_k =0\\
    \implies & \sum_{l=N'+1}^{N} \underbrace{[J(q)]_{γl} ε_l }_{=(J \symbfit{ε})_l} = -  \sum_{k=1}^{N'} \diffp{C_γ(q)}{q_k} ε_k\\
    \implies & (J^{-1} J \symbfit{ε} )_l = ε_l = - \left( \sum_{γ=1}^{Γ}[J^{-1}]_{lγ}  \right) \left(  \sum_{k=1}^{N'} \diffp{C_γ(q)}{q_k} ε_k \right)\\
    \implies & ε_l = - \sum_{γ=1}^{Γ}  \sum_{k=1}^{N'} [J^{-1}]_{lγ} \diffp{C_γ(q)}{q_k} ε_k
  \end{align*}

\end{qboxok}

\begin{qboxok}
  {p63-p64 ラグランジュの未定乗数法の導出1}
  {全員}
  {昌詩,蟹江}

  結局この導出で何を証明しているのかよくわからない.

  \qsep

  拘束条件は式(3.7)で自動的に満たされるから \( \tilde{S}(q,λ) \)の独立な
  \( N + Γ \)  個の変数のうち \( Γ \) 個が消去できる.
  導出でしていることは,ラグランジュの未定乗数法は残りの \( N \) 個の独立変数のうち
  \( Γ \)  個を消去して 
  \( N' \)個(自由度の数)の停留値条件を解くことに帰結できているということである.

  具体的に証明の気持ちを説明する.
  式(3.19)で \( ε_l \) の \( ε_k \) による表現がわかったことで,
  一般に非0であることが判明する.
  そのため, \( i= N'+1,N'+2, \ldots,  N\)でも式(3.6)を満たす必要がある.
  もちろん \( ε_k \)が一般に非0であるため,式(3.6)はすべての\( i \)で満たされなければいけない.
  式(3.6)での未知変数の数は \( N \)   個で式(3.21)で 計 \( Γ \)個の \( λ_γ \) が消去できることがわかるので,
  式(3.22)で自由度の数だけの停留値条件に帰結することができている.

\end{qboxok}




\begin{qboxok}
  {p145 6.1.1 ラグランジアンの凸性}
  {蟹江}
  {蟹江}

  ラグランジアンからハミルトニアンの変換は,ルジャンドル変換になっているが,
  ラグランジアンが正準運動量に関して凸関数になっているという
  のは何から言えるんだろうか.

  \qsep 

  古典力学のラグランジアンに関しては
  \begin{equation}
    L(q, \dot{q},t)=K-U= \frac{1}{2}m \dot{q}^2 -U(q)
    \label{kotenlag}
  \end{equation}
  であるから,
  \[ \diffp[2]L{\dot{q}} = m >0 \]
  となり下に凸であることが保証されている.
  ただし,式\eqref{kotenlag}は直交座標であることを仮定していたりとガバ論理なのでこの回答には修正が必要.
\end{qboxok}

\end{document}