\documentclass{jlreq}
\usepackage{graphicx,tcolorbox,diffcoeff}
\usepackage{amsmath,amsthm}
\usepackage{unicode-math}
\setmathfont{NewComputerModernMath}
\usepackage[thicklines]{cancel}
\usepackage{hyperref}
%使いたいパッケージは競合してエラー置きない限り自由に追加していいよ
\tcbuselibrary{breakable, skins}
\newcommand{\qsep}{%
  \tcblower
  \medskip
}


\newtcolorbox{qbox}[2]{%
  breakable,
  enhanced,
  colback=white,
  colframe=red,
  boxrule=0.8pt,
  arc=2mm,
  left=4mm,
  right=4mm,
  top=3mm,
  bottom=3mm,
  before upper={
    \textbf{#1}
    \medskip
  },
  after upper={
    \par\smallskip
    \hfill{\small 質問者:#2}
  }
}

\newtcolorbox{qboxok}[3]{%
  breakable,
  enhanced,
  colback=white,
  colframe=black,
  boxrule=0.8pt,
  arc=2mm,
  left=4mm,
  right=4mm,
  top=3mm,
  bottom=3mm,
  before upper={
    \textbf{#1}
    \medskip
  },
  after upper={
    \par\smallskip
    \hfill{\small 質問者:#2}
  },
  after lower={
    \par\smallskip
    \hfill{\small 回答者:#3}
  }
}


\begin{document}
% 以下のように書いてください

%--解決していない質問--
% \begin{qbox}
%   {教科書該当箇所}
%   {質問者}
  
%   質問文…
% \end{qbox}

%--解決している質問--
% \begin{qboxok}
%   {教科書該当箇所またはタイトル}
%   {質問者}
%   {回答者}
  
%   質問文…

%   \qsep

%   回答文…
% \end{qboxok}


\begin{qboxok}
  {p29  \( t \)に陽に依存しないラグランジアン}
  {蟹江}
  {蟹江}
  
  「ラグランジアンが\( t \)に陽に依存しない」ってどういう状態かよくわからない.

  \qsep

  \( q , \dot{q}\)が定まったときに,ラグランジアンが時間によらず一定の値を取るということ.
  \( q, \dot{q} \) は測定できるから,二次関数のような具体的な関数形が与えられていなくても
  陽に依存していることを定義できる.

\end{qboxok}

\begin{qboxok}
  {最速降下曲線の変形}
  {まさし(ゼミ中)}
  {蟹江}

  \( p=\tan θ/2 \)という変形をしているが,
  \( p= \difs yx \)なのに勝手にそんなことしていいのか. 

  \qsep

  \( x ,y\)はそれぞれ \( p \)の関数として導くことができました.
  この時点で \( p \)が \( p= \difs yx \)という表式を持っているだけで,媒介変数表示になっており,サイクロイドを表せているわけです.
  たまたま媒介変数表示のパラメーターにわかりやすい性質があっただけで,媒介変数であることに変わりはないので,
  \( p \)の動く範囲がどうであっても全単射な関係を与えられる \( p=\tan θ/2 \)という変換をしていいのです.
\end{qboxok}

\begin{qboxok}
  {p145 6.1.1 ラグランジアンの凸性}
  {蟹江}
  {かにちゃん}

  ラグランジアンからハミルトニアンの変換は,ルジャンドル変換になっているが,
  ラグランジアンが正準運動量に関して凸関数になっているという
  のは何から言えるんだろうか.

  \qsep 

  古典力学のラグランジアンに関しては
  \begin{equation}
    L(q, \dot{q},t)=K-U= \frac{1}{2}m \dot{q}^2 -U(q)
    \label{kotenlag}
  \end{equation}
  であるから,
  \[ \diffp[2]L{\dot{q}} = m >0 \]
  となり下に凸であることが保証されている.
  ただし,式\eqref{kotenlag}は直交座標であることを仮定していたりとガバ論理なのでこの回答には修正が必要.
\end{qboxok}

\end{document}