\documentclass{beamer}
\usepackage{luatexja}
\usetheme[block=fill]{metropolis}
\usepackage{graphicx,tcolorbox,diffcoeff}
\usepackage{amsmath,amsthm}
\usepackage{unicode-math}
\setmathfont{NewComputerModernMath}
\usepackage[thicklines]{cancel}
\usepackage{xcolor}
\renewcommand{\CancelColor}{\color{red!60!black}}

\difdef { f, fp, s, sp } { | }
{
outer-Ldelim
= \left . ,
outer-Rdelim = \right |,
sub-nudge
= 0 mu
}
\usepackage{hyperref}
\title{解析力学自主ゼミ第一回\\
 ~ニュートン力学と変分法~}
\author{蟹江}
\date{2026年2月7日}
\begin{document}

\begin{frame}
  \maketitle
\end{frame}


\begin{frame}{作用・反作用の法則}
  \begin{block}{強い意味での作用・反作用の法則}
    \begin{equation}
      \symbfit{f} _{ij}(t)=-\symbfit{f} _{ji}(t) \propto \symbfit{r} _{i}(t)-\symbfit{r} _j (t) 
      \label{hansayou}
    \end{equation}
  \end{block}
  \vspace{1\baselineskip}
  比例まで含めると\alert{強い意味}での作用・反作用の法則で、\\
  適当なポテンシャルを考えるときに成立しない。
\end{frame}

\begin{frame}{運動量保存則}
  \( M \)個の質点が運動する系で、EoMは
  \begin{equation}
    m_i \ddot{\symbfit{r}}_i(t) = \symbfit{f} _i (t) - \sum_{j=1}^{M} \symbfit{f} _{ij}(t)  
  \end{equation}
  式\eqref{hansayou}より
  \begin{equation}
    \diff**t{\sum_{i=1}^{M} m_i \dot{\symbfit{r}}_i(t) = \sum_{i,j=1}^{M} \symbfit{f} _{ij}(t)= \frac{1}{2} \sum_{i,j=1}^{M} \big(\underbrace{\symbfit{f} _{ij}(t)+\symbfit{f} _{ji}(t)  }_{=0}\big) =0  }
  \end{equation}
  よって運動量 \( \symbfit{P} :=\sum_{i=1}^{M} m_i \dot{\symbfit{r}}_i(t) \) 時間によらず一定値をとる\\
  ただしこのとき、強い作用・反作用の法則は使っていない
\end{frame}

\begin{frame}{角運動量保存則}
  \( \symbfit{f} _i(t) \)が、
  \begin{equation}
    \symbfit{f} _i(t)= C_i(t) \symbfit{r} _i(t) + \sum_{j=1}^{M} \symbfit{f} _{ij}(t)
    \label{fikatei} 
  \end{equation} 
  であることを仮定する。
\end{frame}

\begin{frame}{角運動量保存則の導出}
    \begin{align*}
    \diff{\symbfit{L} }{t}
    =&\diff**t{\sum_{i=1}^{M} m_i \symbfit{r} _i(t) \times \dot{\symbfit{r}}_i(t)}\\
    =& \sum_{i=1}^{M} m_i \underbrace{ \dot{\symbfit{r}}_i(t) \times \dot{\symbfit{r}}_i(t)}_{=0} + \sum_{i=1}^{M} m_i \symbfit{r} _i(t) \times \ddot{\symbfit{r}}_i(t)\\
    =& \sum_{i=1}^{M} \symbfit{r} _i(t) \times \underbrace{m_i \ddot{\symbfit{r} }_i(t)}_{=\symbfit{f} _i(t)} 
  \end{align*}
    式\eqref{fikatei}で \( C_i(t) \symbfit{r} _i(t)  \) の項は \( \symbfit{r} _i(t) \) との外積で0になるため
\end{frame}

\begin{frame}{角運動量保存則の導出}
  \begin{align*}
    \diff{\symbfit{L}}{t}
    =& \sum_{i=1}^{M} \symbfit{r} _i(t) \times \underbrace{m_i \ddot{\symbfit{r} }_i(t)}_{=\symbfit{f} _i(t)} \\
    =& \sum_{i,j=1}^{M} \symbfit{r} _i(t) \times \symbfit{f} _{ij}(t) \\
    =& \sum_{i,j=1}^{M} \frac{1}{2} (\symbfit{r} _i -\symbfit{r} _j)\times \underbrace{\symbfit{f} _{ij}(t) }_{\propto (\symbfit{r} _i-\symbfit{r} _j)}  \\
    =&0
  \end{align*}
  強い意味での作用・反作用の法則を用いている。
\end{frame}

\begin{frame}{関数の停留値問題}
  1変数関数 \( S(q) \)について\alert{停留点}とは、
  \begin{equation}
    δS:=S(q+ε)-S(q)= 𝒪(ε^2)
  \end{equation}
  を満たす\( q \)のことである。
  \uncover<2->{
    \begin{equation}
      S(q+ε)= S(q)+ ε \diff{S(q)}{q} +\frac{1}{2}ε^2 \diff[2]{S(q)}{q}+\cdots 
      \label{ititaylor}
    \end{equation}
  }
  \uncover<2->{式\eqref{ititaylor}より \( \alert{\difs{S(q)}{q}=0} \)のときに停留点となることがわかる }
\end{frame}

\begin{frame}{多変数}
  \( S(q):=S(q_1, \ldots, q_N) \)について停留点を考えるとき、
  \begin{equation}
    δS(q)= \sum_{i=1}^{N} ε_i \diffp{S(q)}{q_i} + \frac{1}{2} \sum_{i,j=1}^{N} ε_i ε_j \diffp{S(q)}{q_i,q_j}+\cdots
    \label{tahentaylo}  
  \end{equation}

  \vspace{1\baselineskip}
  \uncover<2->{式\eqref{tahentaylo}より停留点を取るときの必要十分条件をは、
  \begin{equation}
    \diffp{S(q)}{q_i}=0 \quad (\forall i=1,2, \ldots, N)
  \end{equation}}
\end{frame}

\begin{frame}{汎関数}
  関数\( q \)を1つ定めると値が対応する、\alert{汎関数} \(S[q] \) について考える
  \begin{equation}
    S[q]=\int_{t_i}^{t_f} \dl{t} L(q(t), \dot{q}(t),t) 
  \end{equation}
  この汎関数は \alert{作用}と呼ばれている。
\end{frame}

\begin{frame}
  \begin{center}
    \Huge \textbf{汎関数も停留を考えたい!}
  \end{center}
\end{frame}

\begin{frame}{汎関数の停留(ざっくり)}
  \begin{align*}
  δS[q]
  :=&S[q+ε]+S[q]\\
  =&\int_{t_i}^{t_f} \dl{t}
    (L(q+ε, \dot{q}+ \dot{ε},t)-L(q, \dot{q},t))\\ 
  =&\int_{t_i}^{t_f} \dl{t}
    \Bigl[
      ε\diffp{L}{q}+\dot{ε}\diffp{L}{\dot{q}}
    \Bigr]+ 𝒪(ε ^2)\\
  =&\int_{t_i}^{t_f} \dl{t} \, ε
    \Biggl[
      \diffp{L}{q} - \diff**t{\diffp{L}{\dot{q}}[]}
    \Biggr]+ \only<1>{ε} \only<2>{\xcancel{ε}} \diffp{L}{ \dot{q}}\bigg|_{t_i}^{t_f}  + 𝒪(ε ^2)
\end{align*}
\end{frame}

\begin{frame} 
  \begin{block}{オイラー・ラグランジュ方程式(Euler-Lagrange equation)}
    \( ε(t_i)=ε(t_f)=0 \)となる変分に対して作用 \( S[q] \)が停留となる条件は
    \begin{equation}
      \diff**{t}{\diffp{L(q(t), \dot{q}(t),t)}{ \dot{q}_i(t)}} = \diffp{L(q(t), \dot{q}(t),t)}{q_i(t)}\quad (\forall i=1,2, \ldots, N)
      \label{ELeq}
    \end{equation} 
  \end{block}
   雑に書けば\( \difc{ \difcp{L}{\dot{q}}}{t} = \difcp{L}{q}\)
\end{frame}

\begin{frame}{オイラーラグランジュ方程式の導出}
  \begin{align*}
  δS[q]
  :=&S[q+ε]-S[q]\\
  =&\int_{t_i}^{t_f} \dl{t}
    \big[L(q(t)+ε(t), \dot{q}(t)+ \dot{ε}(t),t)-L(q(t), \dot{q}(t),t)\big]\\ 
  =&\int_{t_i}^{t_f} \dl{t}
    \Biggl[
      ε(t) \diffp.|.{L(q,\dot{q}(t),t)}{q}[q=q(t)]
      + \underbrace{\dot{ε}(t) \diffp.|.{L(q(t),\dot{q},t)}{\dot{q}}[\dot{q}=\dot{q}(t)]}_{\text{ここを部分積分}} 
    \Biggr]\\
    &+ 𝒪(ε ^2)\\
    \uncover<2->{=&\int_{t_i}^{t_f} \dl{t} \, ε(t)
    \Biggl[
      \diffp{L(q(t),\dot{q}(t),t)}{q(t)} - \diff**t{\diffp{L(q(t),\dot{q}(t),t)}{\dot{q}(t)}[]}
    \Biggr]\\
    &+\xcancel{ε(t_f)}\diffp.|.{L(q(t),\dot{q}(t),t)}{ \dot{q}(t)}[t=t_f] - \xcancel{ε(t_i)} \diffp.|.{L(q(t),\dot{q}(t),t)}{ \dot{q}(t)}[t=t_i]\\
    &+ 𝒪(ε ^2)}
\end{align*}
\end{frame}
 

\begin{frame}
  \frametitle{オイラーラグランジュ方程式の導出}
  \( ε(t_i)=ε(t_f)=0 \)を仮定している。
  \( S[q] \)が停留となる条件は、
  \begin{equation}
    \diff**{t}{\diffp{L(q(t), \dot{q}(t),t)}{ \dot{q}(t)}} = \diffp{L(q(t), \dot{q}(t),t)}{q(t)}
  \end{equation}
  となる。
\end{frame}





\begin{frame}{\( 𝒪(ε^2)\)についての補足}
  教科書だけでは分かりづらいため、具体例を示す。\\
  \( g(t_i)=g(t_f)=0 \)を満たす関数 \( g(t) \)  をとって
  \( ε(t)= ηg(t) \)としたとき
  \begin{align*}
  &\int \dl{t} (ε ^2(t)F_1(t) + ε ^3(t)F_2(t)+\cdots ) \\
  =&\int \dl{t} (η ^2g(t)F_1(t) + η ^3g(t)(t)F_2(t)+\cdots )\\ 
  =&η ^2\int \dl{t} (g(t)F_1(t) + η g(t)F_2(t)+\cdots )\\ 
  =& 𝒪(η ^2)
  \end{align*}
  \uncover<2->{微分積分の\alert{線形性}から\\
                先程の雑なオーダーが使えている事がわかる}
\end{frame}

\begin{frame}{\( L(q, \dot{q}, \ddot{q},t) \)のEuler-Lagrange equation}
  \begin{align*}
  δS
  =&\int_{t_i}^{t_f} \dl{t}
    \left( 
      \diffp{L}{q} ε + \diffp{L}{\dot{q}} \dot{ε} + \diffp{L}{ \ddot{q}} \ddot{ε} 
    \right)+ 𝒪(ε^2)\\
  =&\int_{t_i}^{t_f} \dl{t} 
    \left( 
      \diffp{L}{q}ε- \diff**t{\diffp{L}{\dot{q}}[]}ε-\diff**t{\diffp{L}{\ddot{q}}[]} \dot{ε}
     \right)\\
     & + ε\diffp{L}{\dot{q}}\bigg|_{t_i}^{t_f} + \dot{ε} \diffp{L}{\ddot{q}}\bigg|_{t_i}^{t_f} + 𝒪(ε^2)\\
  =&\int_{t_i}^{t_f} \dl{t} ε
    \left( 
      \diffp{L}{q}- \diff**t{\diffp{L}{\dot{q}}[]}+\diff**[2]t{\diffp{L}{\ddot{q}}[]} 
     \right)\\
     &+ ε\diffp{L}{\dot{q}}\bigg|_{t_i}^{t_f}
      + \dot{ε} \diffp{L}{\ddot{q}}\bigg|_{t_i}^{t_f} - ε \diff**t{\diffp{L}{\ddot{q}}[]} \bigg|_{t_i}^{t_f}
      +𝒪(ε^2)
  \end{align*}
\end{frame}
 
\begin{frame}
  \begin{block}{Euler-Lagrange equation 引数 \( (q, \dot{q}, \ddot{q},t) \)}
   \( ε(t_i)=ε(t_f)=0 ,\dot{ε}(t_i)= \dot{ε}(t_f)=0 \)となる変分に対して作用 \( S[q] \)が停留となる条件は
  \begin{equation}
    \diffp{L}{q}- \diff**t{\diffp{L}{\dot{q}}[]}+\diff**[2]t{\diffp{L}{\ddot{q}}[]}=0 
  \end{equation} 
  \end{block}
\end{frame}

\begin{frame}{\( L(q, \dot{q}, \ldots, \overset{n}{q} ,t) \)のEuler-Lagrange equation}
  \begin{align*}
  δS
  =&\int_{t_i}^{t_f} 
    \sum_{i=0}^{n} \diffp{L}{\overset{i}{q} } \overset{i}{ε}  \dl{t} + 𝒪(ε^2)\\
  =&\int_{t_i}^{t_f}ε
    \left( \sum_{i=0}^{n} (-1)^i \diff**[i]t{ \diffp{L}{\overset{i}{q}}[]}  \right) \dl{t} \\
    &+\only<1>{ \sum_{i=1}^{n} \sum_{j=1}^{i}(-1)^{j+1} \overset{i-j}{ε} \diff**[j-1]t{\diffp{L}{ \overset{i}{q}}[]} \Bigg|_{t_i}^{t_f}} \only<2>{\cancel{ \sum_{i=1}^{n} \sum_{j=1}^{i}(-1)^{j+1} \overset{i-j}{ε} \diff**[j-1]t{\diffp{L}{ \overset{i}{q}}[]} \Bigg|_{t_i}^{t_f}}}+ 𝒪(ε^2) 
  \end{align*}
    あんまりに煩雑だったので、時間微分のドットを \( i \)階微分のとき \( \overset{i}{q}\)  のように表した。表記ブレブレでごめんね
\end{frame}

\begin{frame}{一般のEuler-Lagrange equation}
     \( ε(t_i)=ε(t_f)=0 , \dot{ε}(t_i)= \dot{ε}(t_f)=0 ,\ldots, \overset{n-1}{ε}(t_i)=\overset{n-1}{ε}(t_f)=0  \)
   となる変分に対して、作用 \( S[q] \)が停留となる条件は 
  \begin{equation}
    \sum_{i=0}^{n} (-1)^i \diff**[i]t{ \diffp{L}{ \diff[i]{q}{t}}[]} =0
  \end{equation}
\end{frame}

\begin{frame}{エネルギーの定義}
  \begin{block}{エネルギー}
    ラグランジアンが \( L(q, \dot{q},t) \) で与えられる系のエネルギーは
    \begin{equation}
      E(t):=\sum_{i=1}^{n} \dot{q}_i \diffp{L}{ \dot{q}_i} - L 
    \end{equation}
    と定義される。
  \end{block}

\end{frame}

\begin{frame}{エネルギーが保存する条件}
  \begin{align*}
    \diff{E(t)}{t}
    =&\sum_{i=1}^{N} \left( \dot{q}_i \diff**t{ \diffp{L}{ \dot{q}_i}} \only<1>{+\ddot{q}_i(t) \diffp{L}{ \dot{q}_i} } \only<2>{ +\cancel{\ddot{q}_i(t) \diffp{L}{ \dot{q}_i}  }}\right)\only<1-2>{\\}
    \only<1-2>{&}-\sum_{i=1}^{N} \left( \dot{q}_i \diffp{L}{q_i} \only<1>{+\ddot{q}_i(t) \diffp{L}{ \dot{q}_i} } \only<2>{ +\cancel{\ddot{q}_i(t) \diffp{L}{ \dot{q}_i} }}\right)- \diffp{L}{t} \\
    \uncover<3->{=& \sum_{i=1}^{N} \dot{q}_i  \Bigg(\underbrace{\diff**{t}{\diffp{L(q, \dot{q},t)}{ \dot{q}}} - \diffp{L(q,\dot{q},t)}{q}}_{\text{Euler-Lagrange equation}}\Bigg)- \diffp{L}{t}\\ }
    \uncover<4->{=& -\diffp{L}{t}}
  \end{align*} 
  \uncover<5->{ラグランジアンが\alert{陽に}時間依存しないとき\\
                エネルギーは保存される。}
\end{frame}

\begin{frame}{ある座標に陽に依存しない場合の保存量}
  \( p_i(t):= \difsp{L}{\dot{q}_i} \)を定義したとき
  \begin{equation}
    \diff**t{} p_i(t) = \diff**t{ \diffp{L}{ \dot{q}_i}[]} = \diffp{L}{q_i} \quad (\because \text{Euler-Lagrange equation})
  \end{equation} 
  \uncover<2->{ラグランジアンがある座標 \( q_i \)に陽に依存しないとき、\\
                対応する \( p_i \)(\alert{循環座標})が保存される。 } 
\end{frame}

\begin{frame}{ラグランジアンの存在条件}
  式\eqref{ELeq}の時間微分を実行すると
  \begin{equation}
    \mathcal{D}_i:=\sum_{i=j}^{N} \diffp{L}{ \dot{q}_j,\dot{q}_i} \ddot{q}_j +\sum_{j=1}^{N} \diffp{L}{q_j, \dot{q}_i} \dot{q}_j + \diffp{L}{t, \dot{q}_i} - \diffp{L}{q_i}=0
  \end{equation}
  という微分方程式になる

  \vspace{1\baselineskip}
  \uncover<2->{この微分方程式の解はいつでも存在する...?}
\end{frame}

\begin{frame}{ラグランジアンの存在条件}
  \begin{block}{ヘルムホルツ条件}
    微分方程式 \( \mathcal{D}_i(q, \dot{q}, \ddot{q},t) =0 \,(i=1,2, \ldots, N)\)が与えられたとき、
    これを導くラグランジアンが存在するための必要十分条件は、すべての
    \( i,j=1,2, \ldots, N \)  に対して次の関係が成立することである。
    \begin{gather}
      \diffp{\mathcal{D}_i}{ \ddot{q}_j}=\diffp{\mathcal{D}_j}{ \ddot{q}_i}\\
      \frac{1}{2}\left( \diffp{\mathcal{D}_i}{ \dot{q}_j}+ \diffp{\mathcal{D}_j}{ \dot{q}_i}\right) = \diff**t{ \diffp{\mathcal{D}_i}{ \ddot{q}_j}}\\
      \diffp{\mathcal{D}_i}{q_j} - \diffp{\mathcal{D}_j}{q_i}= \frac{1}{2} \diff**t{ \left( \diffp{\mathcal{D}_i}{ \dot{q}_j}-\diffp{\mathcal{D}_j}{ \dot{q}_i} \right)} 
    \end{gather}
  \end{block}
\end{frame}
\end{document}