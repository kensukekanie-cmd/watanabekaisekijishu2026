\documentclass{jlreq}
\usepackage{graphicx,tcolorbox,diffcoeff}
\usepackage{amsmath,amsthm}
\usepackage{unicode-math}
\setmathfont{NewComputerModernMath}
\usepackage[thicklines]{cancel}
\usepackage{xcolor}
\renewcommand{\CancelColor}{\color{red!60!black}}
\newtcolorbox{teiri}[1]{
  colframe=black,
  colback=white,
  coltitle=black,
  colbacktitle=cyan!40!black,
  boxrule=0.8pt,
  arc=0mm,
  title={\textcolor{white}{\textbf{#1}}},
}

\difdef { f, fp, s, sp } { | }
{
outer-Ldelim
= \left . ,
outer-Rdelim = \right |,
sub-nudge
= 0 mu
}


\begin{document}

\begin{align*}
  δS[q]
  :=&S[q+ε]+S[q]\\
  =&\int_{t_i}^{t_f} \dl{t}
    \big[L(q(t)+ε(t), \dot{q}(t)+ \dot{ε}(t),t)-L(q(t), \dot{q}(t),t)\big]\\ 
  =&\int_{t_i}^{t_f} \dl{t}
    \Biggl[
      ε(t) \diffp.|.{L(q,\dot{q}(t),t)}{q}[q=q(t)]
      + \underbrace{\dot{ε}(t) \diffp.|.{L(q(t),\dot{q},t)}{\dot{q}}[\dot{q}=\dot{q}(t)]}_{\text{ここを部分積分}} 
    \Biggr]+ 𝒪(ε ^2)\\
  =&\int_{t_i}^{t_f} \dl{t} \, ε(t)
    \Biggl[
      \diffp{L(q(t),\dot{q}(t),t)}{q(t)} - \diff**t{\diffp{L(q(t),\dot{q}(t),t)}{\dot{q}(t)}[]}
    \Biggr]\\
    &+\xcancel{ε(t_f)}\diffp.|.{L(q(t),\dot{q}(t),t)}{ \dot{q}(t)}[t=t_f] - \xcancel{ε(t_i)} \diffp.|.{L(q(t),\dot{q}(t),t)}{ \dot{q}(t)}[t=t_i]+ 𝒪(ε ^2)
\end{align*}

\( ε(t_i)=ε(t_f)=0 \)を仮定している。
\( S[q] \)が停留となる条件は、
\begin{equation}
  \diff**{t}{\diffp{L(q(t), \dot{q}(t),t)}{ \dot{q}(t)}} = \diffp{L(q(t), \dot{q}(t),t)}{q(t)}
\end{equation}
となる。

\begin{teiri}{オイラー・ラグランジュ方程式(Euler-Lshrange equation)}
  \( ε(t_i)=ε(t_f)=0 \)となる変分に対して作用 \( S[q] \)が停留となる条件は
  \begin{equation}
    \diff**{t}{\diffp{L(q(t), \dot{q}(t),t)}{ \dot{q}_i(t)}} = \diffp{L(q(t), \dot{q}(t),t)}{q_i(t)}\quad (\forall i=1,2, \ldots, N)
  \end{equation} 
  \hrulefill

  {\footnotesize 雑に書けば\( \difc{ \difcp{L}{\dot{q}}}{t} = \difcp{L}{q}\) }
\end{teiri} 


\subsection{\( 𝒪(ε ^2)\)についての補足}  
教科書には、 \( g(t_i)=g(t_f)=0 \)を満たす関数 \( g(t) \)  をとって
\( ε(t)= ηg(t) \)としたとき \( 𝒪(η ^2) \) とある。
この例をメモしておく。
\begin{align*}
  &\int \dl{t} (ε ^2(t)F_1(t) + ε ^3(t)F_2(t)+\cdots ) \\
  =&\int \dl{t} (η ^2g(t)F_1(t) + η ^3g(t)(t)F_2(t)+\cdots )\\ 
  =&η ^2\int \dl{t} (g(t)F_1(t) + η g(t)F_2(t)+\cdots )\\ 
  =& 𝒪(η ^2)
\end{align*}
微分積分の線形性から先程の雑なオーダーが使えている事がわかる。 

\begin{align*}
  δS[q]
  :=&S[q+ε]+S[q]\\
  =&\int_{t_i}^{t_f} \dl{t}
    (L(q+ε, \dot{q}+ \dot{ε},t)-L(q, \dot{q},t))\\ 
  =&\int_{t_i}^{t_f} \dl{t}
    \Bigl[
      ε\diffp{L}{q}+\dot{ε}\diffp{L}{\dot{q}}
    \Bigr]+ 𝒪(ε ^2)\\
  =&\int_{t_i}^{t_f} \dl{t} \, ε
    \Biggl[
      \diffp{L}{q} - \diff**t{\diffp{L}{\dot{q}}[]}
    \Biggr]+ 𝒪(ε ^2)
\end{align*}
 


\end{document}