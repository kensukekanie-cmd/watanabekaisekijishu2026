\documentclass{beamer}
\usepackage{luatexja}
%\renewcommand{\kanjifamilydefault}{\gtdefault}
%これは不便が出るまでは有効化しません
%\usetheme{metropolis}
\usetheme[block=fill]{metropolis}
\usepackage{graphicx,tcolorbox,diffcoeff}
\usepackage{amsmath,amsthm}
\usepackage{unicode-math}
\setmathfont{NewComputerModernMath}
\usepackage[thicklines]{cancel}
\usepackage{xcolor}
\renewcommand{\CancelColor}{\color{red!60!black}}
\newtcolorbox{teiri}[1]{
  colframe=black,
  colback=white,
  coltitle=black,
  colbacktitle=cyan!40!black,
  boxrule=0.8pt,
  arc=0mm,
  title={\textcolor{white}{\textbf{#1}}},
}

\difdef { f, fp, s, sp } { | }
{
outer-Ldelim
= \left . ,
outer-Rdelim = \right |,
sub-nudge
= 0 mu
}

\title{解析力学自主ゼミ第一回\\
 ~ニュートン力学と変分法~}
\author{蟹江}
\date{\today}
\begin{document}

\begin{frame}
  \maketitle
\end{frame}

\begin{frame}
  \frametitle{ざっくりとした導出}
  \begin{align*}
  δS[q]
  :=&S[q+ε]+S[q]\\
  =&\int_{t_i}^{t_f} \dl{t}
    (L(q+ε, \dot{q}+ \dot{ε},t)-L(q, \dot{q},t))\\ 
  =&\int_{t_i}^{t_f} \dl{t}
    \Bigl[
      ε\diffp{L}{q}+\dot{ε}\diffp{L}{\dot{q}}
    \Bigr]+ 𝒪(ε ^2)\\
  =&\int_{t_i}^{t_f} \dl{t} \, ε
    \Biggl[
      \diffp{L}{q} - \diff**t{\diffp{L}{\dot{q}}[]}
    \Biggr]+ 𝒪(ε ^2)
\end{align*}
\end{frame}

\begin{frame} 
  \begin{block}{オイラー・ラグランジュ方程式(Euler-Lagrange equation)}
    \( ε(t_i)=ε(t_f)=0 \)となる変分に対して作用 \( S[q] \)が停留となる条件は
    \begin{equation}
      \diff**{t}{\diffp{L(q(t), \dot{q}(t),t)}{ \dot{q}_i(t)}} = \diffp{L(q(t), \dot{q}(t),t)}{q_i(t)}\quad (\forall i=1,2, \ldots, N)
    \end{equation} 
  \end{block}
   雑に書けば\( \difc{ \difcp{L}{\dot{q}}}{t} = \difcp{L}{q}\)
\end{frame}

\begin{frame}
  \frametitle{オイラーラグランジュ方程式の導出}
  \begin{align*}
  δS[q]
  :=&S[q+ε]-S[q]\\
  =&\int_{t_i}^{t_f} \dl{t}
    \big[L(q(t)+ε(t), \dot{q}(t)+ \dot{ε}(t),t)-L(q(t), \dot{q}(t),t)\big]\\ 
  =&\int_{t_i}^{t_f} \dl{t}
    \Biggl[
      ε(t) \diffp.|.{L(q,\dot{q}(t),t)}{q}[q=q(t)]
      + \underbrace{\dot{ε}(t) \diffp.|.{L(q(t),\dot{q},t)}{\dot{q}}[\dot{q}=\dot{q}(t)]}_{\text{ここを部分積分}} 
    \Biggr]\\
    &+ 𝒪(ε ^2)\\
    \uncover<2->{=&\int_{t_i}^{t_f} \dl{t} \, ε(t)
    \Biggl[
      \diffp{L(q(t),\dot{q}(t),t)}{q(t)} - \diff**t{\diffp{L(q(t),\dot{q}(t),t)}{\dot{q}(t)}[]}
    \Biggr]\\
    &+\xcancel{ε(t_f)}\diffp.|.{L(q(t),\dot{q}(t),t)}{ \dot{q}(t)}[t=t_f] - \xcancel{ε(t_i)} \diffp.|.{L(q(t),\dot{q}(t),t)}{ \dot{q}(t)}[t=t_i]\\
    &+ 𝒪(ε ^2)}
\end{align*}
\end{frame}



\begin{frame}
  \frametitle{オイラーラグランジュ方程式の導出}
  \( ε(t_i)=ε(t_f)=0 \)を仮定している。
  \( S[q] \)が停留となる条件は、
  \begin{equation}
    \diff**{t}{\diffp{L(q(t), \dot{q}(t),t)}{ \dot{q}(t)}} = \diffp{L(q(t), \dot{q}(t),t)}{q(t)}
  \end{equation}
  となる。
\end{frame}





\begin{frame}
  \frametitle{\( 𝒪(ε^2)\)についての補足}
  教科書だけでは分かりづらいため、具体例を示す。\\
  \( g(t_i)=g(t_f)=0 \)を満たす関数 \( g(t) \)  をとって
  \( ε(t)= ηg(t) \)としたとき
  \begin{align*}
  &\int \dl{t} (ε ^2(t)F_1(t) + ε ^3(t)F_2(t)+\cdots ) \\
  =&\int \dl{t} (η ^2g(t)F_1(t) + η ^3g(t)(t)F_2(t)+\cdots )\\ 
  =&η ^2\int \dl{t} (g(t)F_1(t) + η g(t)F_2(t)+\cdots )\\ 
  =& 𝒪(η ^2)
  \end{align*}
  \uncover<2->{微分積分の線形性から先程の雑なオーダーが使えている事がわかる}
\end{frame}

\begin{frame}
  \frametitle{ \( L(q, \dot{q}, \ddot{q},t) \)のEuler-Lagrange equation}
  \begin{align*}
  δS
  =&\int_{t_i}^{t_f} \dl{t}
    \left( 
      \diffp{L}{q} ε + \diffp{L}{\dot{q}} \dot{ε} + \diffp{L}{q} \ddot{ε} 
    \right)+ 𝒪(ε^2)\\
  =&\int_{t_i}^{t_f} \dl{t} 
    \left( 
      \diffp{L}{q}ε- \diff**t{\diffp{L}{\dot{q}}[]}ε-\diff**t{\diffp{L}{\ddot{q}}[]} \dot{ε}
     \right)\\
     & + ε\diffp{L}{\dot{q}}\bigg|_{t_i}^{t_f} + \dot{ε} \diffp{L}{\ddot{q}}\bigg|_{t_i}^{t_f} + 𝒪(ε^2)\\
  =&\int_{t_i}^{t_f} \dl{t} ε
    \left( 
      \diffp{L}{q}- \diff**t{\diffp{L}{\dot{q}}[]}+\diff**[2]t{\diffp{L}{\ddot{q}}[]} 
     \right)\\
     &+ ε\diffp{L}{\dot{q}}\bigg|_{t_i}^{t_f}
      + \dot{ε} \diffp{L}{\ddot{q}}\bigg|_{t_i}^{t_f} - ε \diff**t{\diffp{L}{\ddot{q}}[]} \bigg|_{t_i}^{t_f}
      +𝒪(ε^2)
  \end{align*}
\end{frame}
 
\begin{frame}
  \begin{block}{Euler-Lagrange equation 引数 \( (q, \dot{q}, \ddot{q},t) \)}
   \( ε(t_i)=ε(t_f)=0 ,\dot{ε}(t_i)= \dot{ε}(t_f)=0 \)となる変分に対して作用 \( S[q] \)が停留となる条件は
  \begin{equation}
    \diffp{L}{q}- \diff**t{\diffp{L}{\dot{q}}[]}+\diff**[2]t{\diffp{L}{\ddot{q}}[]}=0 
  \end{equation} 
  \end{block}
\end{frame}

\begin{frame}
  \frametitle{\( L(q, \dot{q}, \ldots, \overset{n}{q} ,t) \)のEuler-Lagrange equation}
  \begin{align*}
  δS
  =&\int_{t_i}^{t_f} 
    \sum_{i=0}^{n} \diffp{L}{\overset{i}{q} } \overset{i}{ε}  \dl{t} + 𝒪(ε^2)\\
  =&\int_{t_i}^{t_f}ε
    \left( \sum_{i=0}^{n} (-1)^i \diff**[i]t{ \diffp{L}{\overset{i}{q}}[]}  \right) \dl{t} 
    + \sum_{i=1}^{n} \sum_{j=1}^{i}(-1)^{j+1} \overset{i-j}{ε} \diff**[j-1]t{\diffp{L}{ \overset{i}{q}}[]} \Bigg|_{t_i}^{t_f} + 𝒪(ε^2) 
  \end{align*}
    あんまりに煩雑だったので、時間微分のドットを \( i \)階微分のとき \( \overset{i}{q}\)  のように表した。表記ブレブレでごめんね
\end{frame}

\begin{frame}
  \frametitle{一般のEuler-Lagrange equation}
     \( ε(t_i)=ε(t_f)=0 , \dot{ε}(t_i)= \dot{ε}(t_f)=0 ,\ldots, \overset{n-1}{ε}(t_i)=\overset{n-1}{ε}(t_f)=0  \)
   となる変分に対して、作用 \( S[q] \)が停留となる条件は 
  \begin{equation}
    \sum_{i=0}^{n} (-1)^i \diff**[i]t{ \diffp{L}{ \diff[i]{q}{t}}[]} =0
  \end{equation}
\end{frame}

\begin{frame}
  \begin{align*}
    \exp x= \sum_{n=0}^{\infty} \frac{1}{n!}x^n 
  \end{align*}
\end{frame}

\end{document}