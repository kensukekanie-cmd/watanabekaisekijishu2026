\documentclass{jlreq}
\usepackage{graphicx,tcolorbox}
\usepackage{mathtools,diffcoeff,unicode-math}
\setmathfont{NewComputerModernMath}
\newtcolorbox{hako}[1]{colframe=black,colback=white,coltitle=black,colbacktitle=white,boxrule=0.8pt,arc=0mm,title={\textbf{#1}},}
\usepackage[thicklines]{cancel}
\usepackage{hyperref}
\setlength{\parindent}{0pt}
\begin{document}

\begin{flushleft}
  \LARGE \textbf{ヘルムホルツ条件}
\end{flushleft}

\[\diff*{\diffp{L}{\dot{q}_i}}{t} - \diffp{L}{q_i}\]
オイラー・ラグランジュ方程式の時間常微分を実行する。
\[\mathcal{D}_i:=\ddot{q}_j ∂_{\dot{i},\dot{j}}L  + \dot{q}_j ∂_{\dot{i},j}L + ∂_{\dot{i},t}L - L_i=0\]
\( \mathcal{D}_i = \mathcal{D}_i(q, \dot{q}, \ddot{q},t) \) このような微分方程式に書き換えられる。
このときの解の存在の必要十分条件が、ヘルムホルツ条件であった。

\begin{flushleft}
  \Large \textbf{必要性}
\end{flushleft}

ラグランジアンの存在を仮定し、
\( \mathcal{D}_i:=\ddot{q}_k ∂_{\dot{i},\dot{k}}L  + \dot{q}_k ∂_{\dot{i},k}L + ∂_{\dot{i},t}L - L_i \) 
とする。\\
\noindent
\( ∂_{\ddot{j}}\mathcal{D}_i \to ∂_{\dot{j}}\mathcal{D}_i \to ∂_{j}\mathcal{D}_i \) の順で計算して、必要性を証明する。


\begin{flushleft}
  \large \textbf{B.1}
\end{flushleft}
\[∂_{\ddot{j}}\mathcal{D}_i = ∂_{\dot{i}, \dot{j}} L =∂_{\dot{j}, \dot{i}} L= ∂_{\ddot{i}}\mathcal{D}_j \]

\begin{flushleft}
  \large \textbf{B.2}
\end{flushleft}
\begin{align*}
  ∂_{\dot{j}}\mathcal{D}_i 
  &= \ddot{q}_k ∂_{\dot{i}, \dot{k}, \dot{j}} L+ \dot{q}_k ∂_{\dot{i}, k,\dot{j} }L + ∂_{\dot{i},j}L + ∂_{\dot{i},t, \dot{j}}L -∂_{i, \dot{j}}L\\
  &= (\ddot{q}_k ∂_{\dot{i},\dot{j}, \dot{k}} + \dot{q}_k ∂_{\dot{i},\dot{j},k} + ∂_{\dot{i},\dot{j},t})L
    + (∂_{\dot{i},j} - ∂_{i, \dot{j}})L\\
  &= \underbrace{\diff*{∂_{\dot{i}, \dot{j}}L}{t} }_{\text{対称}}  + \underbrace{(∂_{\dot{i},j} - ∂_{i, \dot{j}})L}_{\text{反対称}} 
\end{align*} 

これより、
\[\frac{1}{2} (∂_{\dot{j}}\mathcal{D}_i + ∂_{\dot{i}}\mathcal{D}_j ) =  \diff*{∂_{\dot{i}, \dot{j}}L}{t} =\]

\begin{flushleft}
  \large \textbf{B.3}
\end{flushleft}
右辺を \( ∂_{\dot{j}}\mathcal{D}_i  \)の反対称性から整理すると、
\[\frac{1}{2} \diff*{(∂_{\dot{j}}\mathcal{D}_i - ∂_{\dot{i}}\mathcal{D}_j)}{t} = \diff*{(∂_{\dot{i},j}L - ∂_{i, \dot{j}}L)}{t}\] 
次に左辺について考える。
\begin{align*}
  ∂_{j}\mathcal{D}_i
  =& \ddot{q}_k ∂_{\dot{i}, \dot{k}, j} L + \dot{q}_k ∂_{\dot{i}, k, j}L + ∂_{\dot{i},t,j}L - ∂_{i,j} L\\
  =& (\ddot{q}_k ∂_{\dot{i}, j, \dot{k}} L + \dot{q}_k ∂_{\dot{i}, j, k}L + ∂_{\dot{i},j,t}L) - ∂_{i,j} L\\
  =& \diff*{∂_{\dot{i},j}L}{t} - \underbrace{∂_{i,j} L}_{\text{対称}} 
\end{align*}

したがって 
\[∂_{j}\mathcal{D}_i - ∂_{i}\mathcal{D}_j =  \diff*{(∂_{\dot{i},j}L - ∂_{i, \dot{j}}L)}{t} = \frac{1}{2} \diff*{(∂_{\dot{j}}\mathcal{D}_i - ∂_{\dot{i}}\mathcal{D}_j)}{t} \]

% \begin{flushleft}
%   \Large \textbf{十分性}
% \end{flushleft}

% \begin{flushleft}
%   \large \textbf{ヘルムホルツ条件の書き換え}
% \end{flushleft}

% (B.2)の右辺の常微分を実行する。
% \[ \diff*{∂_{\ddot{j}}}{t}= \dot{q}_k ∂_{\ddot{j},k}\mathcal{D}_i + \ddot{q}_k ∂_{\ddot{j}, \dot{k}}\mathcal{D}_i + \dddot{q}_k ∂_{\ddot{j}, \ddot{k}}\mathcal{D}_i + ∂_{\ddot{j},t}\mathcal{D}_i\]
% 左辺には \( \dddot{q}_k \)が含まれないから、
% \[∂_{\ddot{k}, \ddot{j}}\mathcal{D}_i=0 \quad (∀ j,k= 1, \ldots, N)\]
% よって
% \( \mathcal{D}_i \)
% は \( \ddot{q}_j \)のたかだか一次式。
% すなわち\( m_{ij}=m_{ij}(q, \dot{q},t) ,f_i=f_i(q, \dot{q},t) \quad (∀ i,j=1, \ldots, N )  \)を用いると、
% \[\mathcal{D}_i(q, \dot{q}, \ddot{q}, t)= \ddot{q}_k m_{ik} f_i \]
% のように書き換えられる。
% ここから(B.1),(B.2),(B.3)を書き換える。
% まず、(B.1)より、
% \[m_{ij}=m_{ji} \]
% また、
% \[∂_{\dot{j}}\mathcal{D}_i= \ddot{q}_k ∂_{\dot{j}}m_{ik} +∂_{\dot{j}}f_i\]
% \begin{gather*}
%   ∂_{\dot{j}}\mathcal{D}_i- ∂_{\dot{i}}\mathcal{D}_j= \ddot{q}_k \underbrace{(∂_{\dot{j}}m_{ik}- ∂_{\dot{i}}m_{jk})}_{=0}  + (∂_{\dot{j}}f_i-∂_{\dot{i}}f_j)\\
%   ∴ ∂_{\dot{j}}m_{ik} = ∂_{\dot{i}}m_{jk}  
% \end{gather*}
% (B.10)と合わせて考えると、 \(∀  i,j,k=1, \ldots, N \)について成立。

% (B.3)の右辺について
% \begin{align*}
%    \diff*{(∂_{\dot{j}}\mathcal{D}_i- ∂_{\dot{i}}\mathcal{D}_j)}{t}
%    =& \diff*{∂_{\dot{j}}f_i-∂_{\dot{i}}f_j}{t}\\
%    =& \ddot{q}_k (∂_{\dot{k}, \dot{j}}f_i- ∂_{k, \dot{j}}f_j) + (∂_{t}+ \dot{q}_k ∂_{k})(∂_{\dot{j}}f_i -∂_{\dot{i}}f_j)
% \end{align*}
% 左辺は(B.9)から
% \[∂_{j}\mathcal{D}_i-∂_{i}\mathcal{D}_j = \ddot{q}_k(∂_{j}m_{ik}- ∂_{i}m_{jk}) + (∂_{j}f_i-∂_{i}f_j)\]
% 両辺の係数を比較すると、
% \begin{flalign*}
%   &∂_{j}m_{ik}-∂_{i}m_{jk} = \frac{1}{2} (∂_{\dot{k}, \dot{j}}f_i- ∂_{k, \dot{j}}f_j)  \quad (∀ i,j,k=1, \ldots, N)\\
%   &∂_{j}f_i-∂_{i}f_j =\frac{1}{2}  (∂_{t}+ \dot{q}_k ∂_{k})(∂_{\dot{j}}f_i -∂_{\dot{i}}f_j)  \quad (∀ i,j=1, \ldots, N) 
% \end{flalign*}

% (B.2)も書き換える。
% \( ∂_{\dot{j}}m_{ik}  \)が対称であることから、
% \[\frac{1}{2}(∂_{\dot{j}}\mathcal{D}_i +∂_{\dot{i}}\mathcal{D}_j)= \ddot{q}_k ∂_{\dot{j}}m_{ik} + \frac{1}{2}(∂_{\dot{j}}f_i+∂_{\dot{i}}f_j)  \]  

% \begin{align*}
%   \diff*{∂_{\ddot{j}}\mathcal{D}_i}{t}
%   =& \diff*{m_{ij}(q, \dot{q},t)}{t}\\
%   =& \dot{q}_k ∂_{k} m_{ij} + \ddot{q}_k ∂_{\dot{k}}m_{ij} + ∂_{t}m_{ij} \\
%   =& \ddot{q}_k  ∂_{\dot{k}}m_{ij} + (∂_{t} + \dot{q}_k ∂_{k})m_{ij}  
% \end{align*}
% よって
% \[\frac{1}{2}(∂_{\dot{j}}f_i+∂_{\dot{i}}f_j)= (∂_{t} + \dot{q}_k ∂_{k})m_{ij}  \]


\end{document}